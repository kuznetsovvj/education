\documentclass[12pt, a4paper]{article}
\usepackage[T2A]{fontenc}
\usepackage[utf8]{inputenc}
\usepackage[russian]{babel}
\usepackage{amsmath, amssymb} % Для математических символов
\usepackage{enumitem} % Для гибкой настройки списков
\usepackage{geometry} % Для настройки полей
\geometry{top=2cm, bottom=2cm, left=2.5cm, right=2cm}

\begin{document}

\section*{Георгий Игоревич Вольфсон <<Делимость с человеческим лицом>>}

\subsection*{Беседа первая. Примеры и контрпримеры.}

\subsubsection*{Упражнения}

\begin{enumerate}[label=\arabic*., wide=0pt, leftmargin=*]
    \item \textbf{Существует ли такая страна, название которой начинается и заканчивается на одну и ту же букву?}

    \textbf{Да}, например, \emph{Япония}.

    \item \textbf{Верно ли, что любая степень числа $7$ оканчивается на $7$, на $9$ или на $3$?}

    \textbf{Нет}, неверно. $7^4 = 2401$ оканчивается на $1$.

    \item \textbf{Существует ли число с суммой цифр $5$, которое делится на $5$?}

    \textbf{Да}, например, $111110$. Сумма цифр: $1+1+1+1+1+0=5$, и число оканчивается на $0$, значит, делится на $5$.

    \item \textbf{Можно ли представить число $20$ в виде суммы одиннадцати натуральных чисел с одинаковой суммой цифр?}

    \textbf{Да}, например:
    \[
    10 + 1 + 1 + 1 + 1 + 1 + 1 + 1 + 1 + 1 + 1.
    \]
    Сумма цифр каждого слагаемого равна $1$ ($1+0=1$ для числа $10$ и, очевидно, $1$ для остальных слагаемых).

    \item \textbf{Существует ли $50$ чисел, сумма которых в семь раз меньше их произведения?}

    \textbf{Решение:}

    Очевидно, что решение из $50$ единиц не подходит, так как их сумма равна $50$, а произведение~--- $1$.

    Пусть среди чисел $48$ единиц и два других числа $a$ и $b$. Тогда их сумма $S = 48 + a + b$, а произведение $P = 1 \cdot a \cdot b = ab$.

    По условию требуется, чтобы $P = 7S$:
    \[
    ab = 7(48 + a + b).
    \]
    Преобразуем уравнение, чтобы сгруппировать слагаемые с $b$:
    \[
    ab - 7b = 336 + 7a,
    \]
    \[
    b(a - 7) = 336 + 7a.
    \]
    Подберём такое целое $a$, чтобы $a-7$ было делителем правой части. Пусть $a = 8$, тогда:
    \[
    b(8 - 7) = 336 + 7 \cdot 8,
    \]
    \[
    b \cdot 1 = 336 + 56,
    \]
    \[
    b = 392.
    \]
    Получили решение: $a=8$, $b=392$.

    \textbf{Ответ:} Да, существует. Например, числа $392,\ 8$ и $48$ единиц. Проверим:
    \[
    \text{Сумма} = 48 \cdot 1 + 8 + 392 = 448, \quad \text{Произведение} = 1^{48} \cdot 8 \cdot 392 = 3136.
    \]
    Действительно, $3136 / 448 = 7$.
\end{enumerate}

\end{document}