\documentclass[12pt, a4paper]{article}
\usepackage[T2A]{fontenc}
\usepackage[utf8]{inputenc}
\usepackage[russian]{babel}
\usepackage{amsmath, amssymb} % Для математических символов
\usepackage{enumitem} % Для гибкой настройки списков
\usepackage{geometry} % Для настройки полей
\geometry{top=2cm, bottom=2cm, left=2.5cm, right=2cm}

\begin{document}

\section*{Георгий Игоревич Вольфсон <<Делимость с человеческим лицом>>}

\subsection*{Беседа первая. Примеры и контрпримеры.}

\subsubsection*{Упражнения}

\begin{enumerate}[label=\arabic*., wide=0pt, leftmargin=*]
    \item \textbf{Существует ли такая страна, название которой начинается и заканчивается на одну и ту же букву?}

    \textbf{Да}, например, \emph{Япония}, \emph{Аргентина} или \emph{Ангола}.

    \item \textbf{Верно ли, что любая степень числа $7$ оканчивается на $7$, на $9$ или на $3$?}

    \textbf{Нет}, неверно. $7^4 = 2401$ оканчивается на $1$.

    \item \textbf{Существует ли число с суммой цифр $5$, которое делится на $5$?}

    \textbf{Да}, например, $111110$. Сумма цифр: $1+1+1+1+1+0=5$, и число оканчивается на $0$, значит, делится на $5$.

    \item \textbf{Можно ли представить число $20$ в виде суммы одиннадцати натуральных чисел с одинаковой суммой цифр?}

    \textbf{Да}, например:
    \[
    10 + 1 + 1 + 1 + 1 + 1 + 1 + 1 + 1 + 1 + 1.
    \]
    Сумма цифр каждого слагаемого равна $1$ ($1+0=1$ для числа $10$ и, очевидно, $1$ для остальных слагаемых).

    \item \textbf{Существует ли $50$ чисел, сумма которых в семь раз меньше их произведения?}

    \textbf{Решение:}

    Очевидно, что решение из $50$ единиц не подходит, так как их сумма равна $50$, а произведение~--- $1$.

    Пусть среди чисел $48$ единиц и два других числа $a$ и $b$. Тогда их сумма $S = 48 + a + b$, а произведение $P = 1 \cdot a \cdot b = ab$.

    По условию требуется, чтобы $P = 7S$:
    \[
    ab = 7(48 + a + b).
    \]
    Преобразуем уравнение, чтобы сгруппировать слагаемые с $b$:
    \[
    ab - 7b = 336 + 7a,
    \]
    \[
    b(a - 7) = 336 + 7a.
    \]
    Подберём такое целое $a$, чтобы $a-7$ было делителем правой части. Пусть $a = 8$, тогда:
    \[
    b(8 - 7) = 336 + 7 \cdot 8,
    \]
    \[
    b \cdot 1 = 336 + 56,
    \]
    \[
    b = 392.
    \]
    Получили решение: $a=8$, $b=392$.

    \textbf{Ответ:} Да, существует. Например, числа $392,\ 8$ и $48$ единиц. Проверим:
    \[
    \text{Сумма} = 48 \cdot 1 + 8 + 392 = 448, \quad \text{Произведение} = 1^{48} \cdot 8 \cdot 392 = 3136.
    \]
    Действительно, $3136 / 448 = 7$.

    \item \textbf{Существуют ли такие пять трехзначных чисел, что для любой пары из них при вычитании из большего числа меньшего возникает необходимость занимать единицу в одном из разрядов?}

    \textbf{Да}, например, числа $109$, $208$, $307$, $406$ и $505$. При вычитании любого меньшего числа из любого большего в разряде единиц потребуется заём.

    \item \textbf{Существует ли натуральное число, которое равно сумме всех своих натуральных делителей, кроме себя?}

    \textbf{Да}, например, число $6$. Его делители (кроме самого числа): $1, 2, 3$. $1 + 2 + 3 = 6$. Такие числа называются совершенными.

    \item \textbf{Может ли в месяце быть а) пять воскресений, б) шесть воскресений?}

    \textbf{а) Да}, например, если месяц из 31 дня начинается с воскресенья, то воскресеньями будут числа: $1, 8, 15, 22, 29$.

    \textbf{б) Нет}. Для шести воскресений в месяце необходимо минимум $5$ полных недель и ещё один день, что составляет $5 \times 7 + 1 = 36$ дней. Поскольку самый длинный месяц имеет $31$ день, $36$ дней в месяце быть не может.

    \item \textbf{Верно ли, что в русском языке нет слова, в котором 5 согласных идут подряд?}

    \textbf{Нет}, неверно. Контрпример: слово <<\emph{контрпр}имер>> (буквы \emph{н, т, р, п, р} идут подряд).

    \item \textbf{В 20-этажном доме, где живет Давид, лифт сломался, и теперь там работают только две кнопки <<+5>> и <<–7>> (первая поднимает лифт на пять этажей вверх, вторая опускает на семь вниз, а если это сделать невозможно, то кнопка просто не срабатывает). Может ли Давид подняться на этом лифте с первого этажа на второй?}

    \textbf{Да}, может. Например, выполнив следующую последовательность нажатий: три раза <<+5>> и два раза <<–7>>.
    \[
    1 \xrightarrow{+5} 6 \xrightarrow{+5} 11 \xrightarrow{+5} 16 \xrightarrow{-7} 9 \xrightarrow{-7} 2.
    \]
    Алгебраически: $3 \times 5 - 2 \times 7 = 15 - 14 = 1$.

    \item \textbf{К числу $366$ добавили произведение его цифр. К полученному числу добавили произведение цифр этого числа и т.д. Можно ли с помощью таких операций получить число $2019$?}

    \textbf{Решение:}
    Проследим за операциями:
    \begin{align*}
    366 &+ (3 \cdot 6 \cdot 6) = 366 + 108 = 474, \\
    474 &+ (4 \cdot 7 \cdot 4) = 474 + 112 = 586, \\
    586 &+ (5 \cdot 8 \cdot 6) = 586 + 240 = 826, \\
    826 &+ (8 \cdot 2 \cdot 6) = 826 + 96 = 922, \\
    922 &+ (9 \cdot 2 \cdot 2) = 922 + 36 = 958, \\
    958 &+ (9 \cdot 5 \cdot 8) = 958 + 360 = 1318, \\
    1318 &+ (1 \cdot 3 \cdot 1 \cdot 8) = 1318 + 24 = 1342, \\
    1342 &+ (1 \cdot 3 \cdot 4 \cdot 2) = 1342 + 24 = 1366, \\
    1366 &+ (1 \cdot 3 \cdot 6 \cdot 6) = 1366 + 108 = 1474.
    \end{align*}
    Можно заметить, что следуещие значения в последовательности $1586$, $1826$, $1922$, $1958$, $2318$ (так как произведения цифр чисел $366$ и $1366$ равны). Так как последовательность только возрастает, значит получить $2019$ нельзя.

    \textbf{Ответ:} Нет, получить число $2019$ с помощью таких операций, начиная с $366$, нельзя.
\end{enumerate}

\end{document}