\documentclass[12pt, a4paper]{article}
\usepackage[T2A]{fontenc}
\usepackage[utf8]{inputenc}
\usepackage[russian]{babel}
\usepackage{amsmath, amssymb} % Для математических символов
\usepackage{enumitem} % Для гибкой настройки списков
\usepackage{geometry} % Для настройки полей
\geometry{top=2cm, bottom=2cm, left=2.5cm, right=2cm}

\begin{document}

\section*{Георгий Игоревич Вольфсон <<Делимость с человеческим лицом>>}

\subsection*{Беседа вторая. Свойства делимости.}

\subsubsection*{Упражнения}

\begin{enumerate}[label=\arabic*., wide=0pt, leftmargin=*]

\item \textbf{Докажите, что $35a + 14b$ кратно 7 при любых натуральных $a$ и $b$.}

\textbf{Решение:} 
$35a$ делится на 7, так как $35 = 7 \times 5$. 
$14b$ делится на 7, так как $14 = 7 \times 2$. 
Следовательно, сумма $35a + 14b$ делится на 7.

\item \textbf{Найдите все натуральные $a$, при которых $a + 2$ делится на $a + 1$.}

\textbf{Решение:}
\[
\frac{a + 2}{a + 1} = 1 + \frac{1}{a + 1}.
\]
Для целости дроби $\frac{1}{a + 1}$ должно быть целым числом. Это возможно только при $a + 1 = 1$, то есть $a = 0$, но тогда $a$ не является натуральным числом.

\textbf{Ответ: } Нет решений.

\item \textbf{Найдите все натуральные $a$, при которых $3a - 2$ делится на $a + 2$.}

\textbf{Решение:}
\[
\frac{3a - 2}{a + 2} = 3 - \frac{8}{a + 2}.
\]
Для целости дроби $\frac{8}{a + 2}$ должно быть целым числом. Следовательно, $a + 2$ должно быть делителем числа 8. Делители 8: $\pm1, \pm2, \pm4, \pm8$. Учитывая, что $a$ — натуральное, получаем:
\begin{align*}
a + 2 &= 1 \Rightarrow a = -1 \quad \text{(не подходит)}, \\
a + 2 &= 2 \Rightarrow a = 0, \quad \text{(не подходит)}\\
a + 2 &= 4 \Rightarrow a = 2, \\
a + 2 &= 8 \Rightarrow a = 6.
\end{align*}

\textbf{Ответ: } $2$ и $6$.

\item \textbf{Найдите все натуральные $a$, при которых $7a + 1$ делится на $2a - 1$.}

\textbf{Решение:}
\[
\frac{7a + 1}{2a - 1} = 3 + \frac{a + 4}{2a - 1}.
\]
Дробь $\frac{a + 4}{2a - 1}$ должна быть целым числом. Заметим, что $2a - 1 \leq a + 4$ при $a \leq 5$. Проверим натуральные значения $a = 1, 2, 3, 4, 5$:
\begin{align*}
a = 1: &\quad \frac{5}{1} = 5 \quad \text{целое}, \\
a = 2: &\quad \frac{6}{3} = 2 \quad \text{целое}, \\
a = 3: &\quad \frac{7}{5} = 1.4 \quad \text{не целое}, \\
a = 4: &\quad \frac{8}{7} \approx 1.14 \quad \text{не целое}, \\
a = 5: &\quad \frac{9}{9} = 1 \quad \text{целое}.
\end{align*}

\textbf{Ответ: } $1, 2, 5$.

\item \textbf{Найдите все целые $a$, при которых $2a + 9$ делится на $a - 1$.}

\textbf{Решение:}
\[
\frac{2a + 9}{a - 1} = 2 + \frac{11}{a - 1}.
\]
Для целости дроби $\frac{11}{a - 1}$ должно быть целым числом. Следовательно, $a - 1$ должно быть делителем числа 11. Делители 11: $\pm1, \pm11$. Получаем:
\begin{align*}
a - 1 &= 1 \Rightarrow a = 2, \\
a - 1 &= -1 \Rightarrow a = 0, \\
a - 1 &= 11 \Rightarrow a = 12, \\
a - 1 &= -11 \Rightarrow a = -10.
\end{align*}

\textbf{Ответ: } $-10, 0, 2, 12$.

\item \textbf{Владимир Анатольевич зашел в магазин и купил семь картриджей для принтера, 35 тетрадей для контрольных работ и несколько пачек бумаги, каждая из которых стоила 280 рублей. Каждый товар стоил целое количество рублей. Кассир назвал общую сумму в 4005 рублей, на что Владимир Анатольевич возразил, что он — учитель математики и умеет считать. Почему он решил, что кассир ошибся?}

\textbf{Решение:}
Пусть $x$ — цена картриджа, $y$ — цена тетради, $z$ — количество пачек бумаги. Тогда:
\[
7x + 35y + 280z = 4005.
\]
Заметим, что левая часть делится на 7, так как каждое слагаемое кратно 7. Но правая часть $4005$ не делится на 7 ($4005 \div 7 = 572.142\ldots$). Следовательно, уравнение не имеет решений в целых числах, и кассир ошибся.

\item \textbf{Докажите, что $29^2 - 13^2$ делится на 16.}

\textbf{Решение:}
По формуле разности квадратов:
\[
29^2 - 13^2 = (29 - 13)(29 + 13) = 16 \times 42.
\]
Так как один из множителей равен 16, всё произведение делится на 16.

\item \textbf{Докажите, что $41^3 + 59^3$ круглое число (то есть оканчивается на ноль).}

\textbf{Решение:}
По формуле суммы кубов:
\[
41^3 + 59^3 = (41 + 59)(41^2 - 41 \cdot 59 + 59^2) = 100 \times (41^2 - 41 \cdot 59 + 59^2).
\]
Так как первый множитель равен 100, произведение оканчивается двумя нулями.

\item \textbf{Найдите все натуральные $a$, при которых $a + 4$ делится на $a^2 - 2$.}

\textbf{Решение:}
Рассмотрим дробь:
\[
\frac{a + 4}{a^2 - 2}.
\]
При $a \geq 4$ знаменатель $a^2 - 2$ становится больше числителя $a + 4$, поэтому проверим малые значения:
\begin{align*}
a = 1: &\quad \frac{5}{-1} = -5 \quad \text{целое}, \\
a = 2: &\quad \frac{6}{2} = 3 \quad \text{целое}, \\
a = 3: &\quad \frac{7}{7} = 1 \quad \text{целое}, \\
a = 4: &\quad \frac{8}{14} \approx 0.57 \quad \text{не целое}.
\end{align*}

\textbf{Ответ: } $1, 2, 3$.
 
\item \textbf{Число $x + 5$ делится на 4. Докажите, что $5x + 13$ также делится на 4.}

\textbf{Решение:}
Преобразуем выражение:
\[
5x + 13 = 5(x + 5) - 12.
\]
По условию $x + 5$ делится на 4, значит $5(x + 5)$ делится на 4. Также $12$ делится на 4. Следовательно, разность $5(x + 5) - 12$ делится на 4.

\item \textbf{Число $24a$ делится на 15. Верно ли, что $a$ делится на 15?}

\textbf{Решение:}
Нет, неверно. Контрпример: $a = 5$. Тогда $24 \times 5 = 120$ делится на 15, но $5$ не делится на 15.

\end{enumerate}

\end{document}