\documentclass[12pt, a4paper]{article}
\usepackage[T2A]{fontenc}
\usepackage[utf8]{inputenc}
\usepackage[russian]{babel}
\usepackage{amsmath, amssymb} % Для математических символов
\usepackage{enumitem} % Для гибкой настройки списков
\usepackage{geometry} % Для настройки полей
\geometry{top=2cm, bottom=2cm, left=2.5cm, right=2cm}

\begin{document}

\section*{Георгий Игоревич Вольфсон <<Делимость с человеческим лицом>>}

\subsection*{Задачник}

\subsubsection*{Задача 1.1}
\textbf{Существуют ли пять различных натуральных чисел, сумма которых делится на каждое из них?}

\textbf{Решение:}
Попробуем найти число, которое равно сумме своих пяти делителей. 

Числа меньше $15$ рассматривать нет смысла ($15 = 1 + 2 + 3 + 4 + 5$). 
\begin{itemize}
    \item У $16$ всего четыре делителя ($1, 2, 4, 8$).
    \item Нечётные числа нам не подойдут из-за малого количества делителей.
    \item У $18$ пять делителей ($1, 2, 3, 6, 9$), но $1 + 2 + 3 + 6 + 9 = 21 \neq 18$.
    \item У $20$ и $22$ не хватает делителей.
    \item У $24$ пять делителей: $1, 2, 3, 4, 6, 8, 12$. Из них можно составить сумму: $1 + 2 + 3 + 6 + 12 = 24$.
\end{itemize}

\textbf{Ответ:} Да, например, $24 = 1 + 2 + 3 + 6 + 12$.

\subsubsection*{Задача 1.2}
\textbf{Существует ли четырехзначное число, которое при делении на $131$ дает остаток $112$, а при делении на $132$ — остаток $98$?}

\textbf{Решение:}
Пусть искомое число — $x$. Тогда:
\[
\begin{cases}
x = 131k_1 + 112, \\
x = 132k_2 + 98,
\end{cases}
\]
где $k_1, k_2 \in \mathbb{Z}$.

Приравняем правые части:
\[
131k_1 + 112 = 132k_2 + 98.
\]
Перенесём слагаемые:
\[
131k_1 - 132k_2 = -14,
\]
\[
131k_1 - 131k_2 - k_2 = -14,
\]
\[
131(k_1 - k_2) - k_2 = -14.
\]
Обозначим $m = k_1 - k_2$, тогда:
\[
131m - k_2 = -14 \quad \Rightarrow \quad k_2 = 131m + 14.
\]

Подставим $k_2$ во второе уравнение:
\[
x = 132(131m + 14) + 98 = 132 \cdot 131m + 132 \cdot 14 + 98 = 17292m + 1848 + 98 = 17292m + 1946.
\]

Найдём четырёхзначное $x$ при $m = 0$:
\[
x = 1946.
\]
Проверим:
\[
1946 = 131 \cdot 14 + 112, \quad 1946 = 132 \cdot 14 + 98.
\]

\textbf{Ответ:} Да, например, $1946$.

\subsubsection*{Задача 1.3}
\textbf{Приведите пример натурального числа, которое при делении на 3, на 4, на 5 и на 7 дает остаток 1.}

\textbf{Решение:}
Если число при делении на данные числа даёт остаток 1, то число минус 1 должно делиться нацело на каждое из них. То есть $x - 1$ должно делиться на $3$, $4$, $5$ и $7$.

Наименьшее число, которое делится на все эти числа — это их наименьшее общее кратное (НОК). Поскольку $3$, $4$, $5$, $7$ — попарно взаимно простые числа, то:
\[
\text{НОК}(3, 4, 5, 7) = 3 \times 4 \times 5 \times 7 = 420.
\]

Тогда искомое число:
\[
x = 420 + 1 = 421.
\]

Проверим:
\begin{align*}
421 : 3 &= 140 \quad (\text{остаток } 1), \\
421 : 4 &= 105 \quad (\text{остаток } 1), \\
421 : 5 &= 84 \quad (\text{остаток } 1), \\
421 : 7 &= 60 \quad (\text{остаток } 1).
\end{align*}

\textbf{Ответ:} Да, например, $421$.

\subsubsection*{Задача 1.4}
\textbf{Приведите пример натурального числа, которое при делении на 3 дает остаток 1, при делении на 5 — остаток 3, при делении на 7 — остаток 5, при делении на 11 — остаток 9.}

\textbf{Решение:}
Найдем число, которое удовлетворяет первым двум требованиям. Числа, дающие при делении на 3 остаток 1: 
\[
1, 4, 7, 10, 13, 16, 19, 22, 25, 28, 31, \ldots
\]
Среди них найдем те, которые при делении на 5 дают остаток 3:
\[
13 \ (\text{т.к. } 13 : 5 = 2 \text{ ост. } 3),\quad 
28 \ (\text{т.к. } 28 : 5 = 5 \text{ ост. } 3),\quad 
43,\ 58,\ \ldots
\]
Получаем арифметическую прогрессию с разностью $15$ (НОК(3,5)=15):
\[
13,\ 28,\ 43,\ 58,\ 73,\ 88,\ 103,\ 118,\ \ldots
\]

Теперь добавим третье условие (остаток 5 при делении на 7). Проверим числа из полученной последовательности:
\begin{align*}
13 : 7 &= 1 \text{ ост. } 6 \quad \text{не подходит} \\
28 : 7 &= 4 \text{ ост. } 0 \quad \text{не подходит} \\
43 : 7 &= 6 \text{ ост. } 1 \quad \text{не подходит} \\
58 : 7 &= 8 \text{ ост. } 2 \quad \text{не подходит} \\
73 : 7 &= 10 \text{ ост. } 3 \quad \text{не подходит} \\
88 : 7 &= 12 \text{ ост. } 4 \quad \text{не подходит} \\
103 : 7 &= 14 \text{ ост. } 5 \quad \text{подходит!}
\end{align*}

Получаем новую последовательность чисел, удовлетворяющих первым трем условиям, с разностью $105$ (НОК(3,5,7)=105):
\[
103,\ 208,\ 313,\ 418,\ 523,\ 628,\ 733,\ 838,\ 943,\ 1048,\ 1153,\ \ldots
\]

Добавим четвертое условие (остаток 9 при делении на 11). Проверим:
\begin{align*}
103 : 11 &= 9 \text{ ост. } 4 \quad \text{не подходит} \\
208 : 11 &= 18 \text{ ост. } 10 \quad \text{не подходит} \\
313 : 11 &= 28 \text{ ост. } 5 \quad \text{не подходит} \\
418 : 11 &= 38 \text{ ост. } 0 \quad \text{не подходит} \\
523 : 11 &= 47 \text{ ост. } 6 \quad \text{не подходит} \\
628 : 11 &= 57 \text{ ост. } 1 \quad \text{не подходит} \\
733 : 11 &= 66 \text{ ост. } 7 \quad \text{не подходит} \\
838 : 11 &= 76 \text{ ост. } 2 \quad \text{не подходит} \\
943 : 11 &= 85 \text{ ост. } 8 \quad \text{не подходит} \\
1048 : 11 &= 95 \text{ ост. } 3 \quad \text{не подходит} \\
1153 : 11 &= 104 \text{ ост. } 9 \quad \text{подходит!}
\end{align*}

\textbf{Ответ:} Да, например, $1153$.

\subsubsection*{Задача 1.5}
\textbf{Можно ли представить число 2021 в виде дроби, числитель которой — девятая степень целого числа, а знаменатель — десятая степень целого числа?}

\textbf{Решение:}
Разложим $2021$ на простые множители: $2021 = 43 \times 47$.

Предположим, что существуют целые числа $a$ и $b$ такие, что:
\[
\frac{a^9}{b^{10}} = 2021.
\]

Тогда:
\[
a^9 = 2021 \cdot b^{10} = 43 \cdot 47 \cdot b^{10}.
\]

По основной теореме арифметики, степени простых чисел в левой и правой части должны совпадать. Пусть $a = 43^{k_1} \cdot 47^{k_2} \cdot c$, где $c$ — некоторое целое число, не делящееся на $43$ и $47$. Тогда:
\[
a^9 = 43^{9k_1} \cdot 47^{9k_2} \cdot c^9.
\]

С другой стороны:
\[
a^9 = 43^1 \cdot 47^1 \cdot b^{10}.
\]

Приравнивая степени простых множителей, получаем систему:
\[
\begin{cases}
9k_1 = 1 + 10m_1, \\
9k_2 = 1 + 10m_2,
\end{cases}
\]
где $m_1$, $m_2$ — степени чисел $43$ и $47$ в разложении $b$.

Найдём натуральные решения первого уравнения: $9k_1 - 10m_1 = 1$. Подбором находим $k_1 = 9$, $m_1 = 8$:
\[
9 \times 9 - 10 \times 8 = 81 - 80 = 1.
\]

Аналогично для второго уравнения: $k_2 = 9$, $m_2 = 8$.

Таким образом, можно взять:
\[
a = 43^9 \cdot 47^9,\quad b = 43^8 \cdot 47^8.
\]

Тогда:
\[
\frac{a^9}{b^{10}} = \frac{(43^9 \cdot 47^9)^9}{(43^8 \cdot 47^8)^{10}} = \frac{43^{81} \cdot 47^{81}}{43^{80} \cdot 47^{80}} = 43 \cdot 47 = 2021.
\]

\textbf{Ответ:} Да, можно.

\subsubsection*{Задача 1.6}
\textbf{Существует ли такое натуральное число, что сумма его цифр больше суммы цифр его квадрата?}

\textbf{Решение:}
Рассмотрим числа, оканчивающиеся на $9$. При возведении в квадрат такого числа происходит "перенос" единицы в старший разряд, что может значительно уменьшить сумму цифр.

Например, возьмём $49$:
\begin{align*}
49^2 &= 2401, \\
S(49) &= 4 + 9 = 13, \\
S(2401) &= 2 + 4 + 0 + 1 = 7.
\end{align*}

Действительно, $13 > 7$.

\textbf{Ответ:} Да, например, $49$.

\subsubsection*{Задача 1.7}
\textbf{Может ли произведение квадрата и куба некоторого натурального числа, большего единицы, быть шестой степенью натурального числа?}

\textbf{Решение:}
Пусть $a$ и $b$ — искомые натуральные числа. Тогда:
\[
a^2 \cdot a^3 = b^6.
\]
Это можно переписать как:
\[
a^5 = b^6.
\]
Отсюда:
\[
b = a^5/b^5 = (a/b)^5.
\]
Значит, $b$ должен быть пятой степенью некоторого числа. Пусть $b = c^5$, тогда $c^5 = (a/c^5)^5$, то есть $a = c^6$.


Возьмём $c = 2$ (натуральное число, большее 1):
\[
a = 2^6 = 64,\quad b = 2^5 = 32.
\]
Проверим:
\[
64^2 \cdot 64^3 = 64^5 = (2^6)^5 = 2^{30},
\]
\[
32^6 = (2^5)^6 = 2^{30}.
\]
Действительно, $2^{30} = 2^{30}$.

\textbf{Ответ:} Да, например, при $a = 64$ получаем $64^2 \cdot 64^3 = 32^6$.

\subsubsection*{Задача 1.8}
\textbf{Используя только цифры 1 и 7 (каждую — не более четырех раз), знаки арифметических действий и скобки, составьте выражение, значение которого равно 2006.}

\textbf{Решение:}
Будет позже.

\subsubsection*{Задача 1.9}
\textbf{Существует ли десятизначное число, делящееся на 11, все цифры которого различны?}

\textbf{Решение:}
Воспользуемся признаком делимости на 11: разность суммы цифр, стоящих на чётных позициях, и суммы цифр, стоящих на нечётных позициях, должна делиться на 11.

Обозначим:
\[
S_{\text{чёт}} - S_{\text{нечёт}} = 11k,\quad \text{где } k \in \mathbb{Z}.
\]

Сумма всех цифр от 0 до 9 равна:
\[
S_{\text{чёт}} + S_{\text{нечёт}} = 0 + 1 + 2 + \cdots + 9 = 45.
\]

Рассмотрим возможные значения $k$:
\begin{itemize}
    \item $k = 0$: тогда $S_{\text{чёт}} = S_{\text{нечёт}} = 22.5$ — невозможно, так как суммы цифр целые.
    \item $k = 1$: тогда $S_{\text{чёт}} - S_{\text{нечёт}} = 11$ и $S_{\text{чёт}} + S_{\text{нечёт}} = 45$. Решая систему, получаем:
    \[
    S_{\text{чёт}} = 28,\quad S_{\text{нечёт}} = 17.
    \]
    \item $k = -1$: $S_{\text{чёт}} = 17$, $S_{\text{нечёт}} = 28$.
    \item $|k| > 1$: невозможно, так как максимальная разность сумм не превышает $25$ ($9+8+7+6+5 - (0+1+2+3+4) = 35-10=25$).
\end{itemize}

Подберём пять цифр с суммой 28 для чётных позиций: $9 + 8 + 6 + 3 + 2 = 28$. Тогда для нечётных позиций остаются цифры: $0, 1, 4, 5, 7$ с суммой $0+1+4+5+7=17$.

Составим число, удовлетворяющее условию: $9\,7\,8\,5\,6\,4\,3\,1\,2\,0$ 

Проверим делимость на 11:
\[
9785643120 \div 11 = 889603920.
\]

\textbf{Ответ:} Да, например, $9785643120$.

\subsubsection*{Задача 1.10}
\textbf{Можно ли поставить в ряд все натуральные числа от 1 до 100 так, чтобы каждые два соседних числа отличались либо на 2, либо в два раза?}

\textbf{Решение:}
Рассмотрим чётные и нечётные числа. Заметим, что:
\begin{itemize}
    \item Все чётные числа можно выстроить в цепочку: $100, 98, 96, \ldots, 4, 2$.
    \item Все нечётные числа также можно выстроить: $1, 3, 5, \ldots, 97, 99$.
\end{itemize}

Теперь соединим эти цепочки. Число 2 (чётное) и число 1 (нечётное) отличаются в 2 раза. Таким образом, получаем последовательность:
\[
100, 98, 96, \ldots, 4, 2, 1, 3, 5, \ldots, 97, 99.
\]

Проверим условия:
\begin{itemize}
    \item В цепочке чётных чисел соседние отличаются на 2.
    \item В цепочке нечётных чисел соседние отличаются на 2.
    \item На стыке цепочек: $2$ и $1$ отличаются в 2 раза.
\end{itemize}

\textbf{Ответ:} Да, например, $100, 98, 96, \ldots, 4, 2, 1, 3, 5, \ldots, 97, 99$.

\end{document}