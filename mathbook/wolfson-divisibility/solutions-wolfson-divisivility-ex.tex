\documentclass[12pt, a4paper]{article}
\usepackage[T2A]{fontenc}
\usepackage[utf8]{inputenc}
\usepackage[russian]{babel}
\usepackage{amsmath, amssymb} % Для математических символов
\usepackage{enumitem} % Для гибкой настройки списков
\usepackage{geometry} % Для настройки полей
\geometry{top=2cm, bottom=2cm, left=2.5cm, right=2cm}

\begin{document}

\section*{Георгий Игоревич Вольфсон <<Делимость с человеческим лицом>>}

\subsection*{Задачник}

\subsubsection*{Задача 1.1}
\textbf{Существуют ли пять различных натуральных чисел, сумма которых делится на каждое из них?}

\textbf{Решение:}
Попробуем найти число, которое равно сумме своих пяти делителей. 

Числа меньше $15$ рассматривать нет смысла ($15 = 1 + 2 + 3 + 4 + 5$). 
\begin{itemize}
    \item У $16$ всего четыре делителя ($1, 2, 4, 8$).
    \item Нечётные числа нам не подойдут из-за малого количества делителей.
    \item У $18$ пять делителей ($1, 2, 3, 6, 9$), но $1 + 2 + 3 + 6 + 9 = 21 \neq 18$.
    \item У $20$ и $22$ не хватает делителей.
    \item У $24$ пять делителей: $1, 2, 3, 4, 6, 8, 12$. Из них можно составить сумму: $1 + 2 + 3 + 6 + 12 = 24$.
\end{itemize}

\textbf{Ответ:} Да, например, $24 = 1 + 2 + 3 + 6 + 12$.

\subsubsection*{Задача 1.2}
\textbf{Существует ли четырехзначное число, которое при делении на $131$ дает остаток $112$, а при делении на $132$ — остаток $98$?}

\textbf{Решение:}
Пусть искомое число — $x$. Тогда:
\[
\begin{cases}
x = 131k_1 + 112, \\
x = 132k_2 + 98,
\end{cases}
\]
где $k_1, k_2 \in \mathbb{Z}$.

Приравняем правые части:
\[
131k_1 + 112 = 132k_2 + 98.
\]
Перенесём слагаемые:
\[
131k_1 - 132k_2 = -14,
\]
\[
131k_1 - 131k_2 - k_2 = -14,
\]
\[
131(k_1 - k_2) - k_2 = -14.
\]
Обозначим $m = k_1 - k_2$, тогда:
\[
131m - k_2 = -14 \quad \Rightarrow \quad k_2 = 131m + 14.
\]

Подставим $k_2$ во второе уравнение:
\[
x = 132(131m + 14) + 98 = 132 \cdot 131m + 132 \cdot 14 + 98 = 17292m + 1848 + 98 = 17292m + 1946.
\]

Найдём четырёхзначное $x$ при $m = 0$:
\[
x = 1946.
\]
Проверим:
\[
1946 = 131 \cdot 14 + 112, \quad 1946 = 132 \cdot 14 + 98.
\]

\textbf{Ответ:} Да, например, $1946$.

\subsubsection*{Задача 1.3}
\textbf{Приведите пример натурального числа, которое при делении на 3, на 4, на 5 и на 7 дает остаток 1.}

\textbf{Решение:}
Если число при делении на данные числа даёт остаток 1, то число минус 1 должно делиться нацело на каждое из них. То есть $x - 1$ должно делиться на $3$, $4$, $5$ и $7$.

Наименьшее число, которое делится на все эти числа — это их наименьшее общее кратное (НОК). Поскольку $3$, $4$, $5$, $7$ — попарно взаимно простые числа, то:
\[
\text{НОК}(3, 4, 5, 7) = 3 \times 4 \times 5 \times 7 = 420.
\]

Тогда искомое число:
\[
x = 420 + 1 = 421.
\]

Проверим:
\begin{align*}
421 : 3 &= 140 \quad (\text{остаток } 1), \\
421 : 4 &= 105 \quad (\text{остаток } 1), \\
421 : 5 &= 84 \quad (\text{остаток } 1), \\
421 : 7 &= 60 \quad (\text{остаток } 1).
\end{align*}

\textbf{Ответ:} Да, например, $421$.

\end{document}